\subsection{JEE beans}\label{subsec:archjee}

JEE beans provide business logic interfaces implementation. They implement the
interface between MySQL database (for user data) and Erlang subsystem towards
servlets. It is composed of the following components:

\begin{description}
	\item[AuctionManager] A singleton session bean that manages the
		communication with the Erlang subsystem, exposing the interfaces
		between the former and servlets. This bean is also in charge of
		communicating changes in the auctions’ state from the Erlang
		subsystem to the \code{AuctionStatePublisher}.
	\item[UserManager] A stateless session bean that manages the
		communication with the MySQL database (that manages user
		informations), exposing the interfaces between the former and
		servlets.
	\item[AuctionStatePublisher] A singleton session bean that act as a
		proxy for the actual state of all the bids in the Erlang
		subsystem, whenever the state changes (because a bid has been
		made or deleted, or an auction is closed) this bean is informed
		from the \code{AuctionManager}. Furthermore, it acts as a
		producer for a JMS queue, to promptly notify clients on changes
		in the auction
\end{description}
